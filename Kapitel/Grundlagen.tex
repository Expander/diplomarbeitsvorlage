\chapter{Grundlagen}

\section{Allgemeine Grundlagen}

Hier ein Feynman-Diagramm zu $(g-2)_\mu$ im MSSM:

\begin{center}
\begin{fmffile}{Feynman/QED+Chargino}
  \fmfset{thin}{.5pt}
  \fmfset{wiggly_len}{3mm}
  \fmfset{dash_len}{2.5mm}
  \fmfset{dot_size}{1thick}
  \fmfset{arrow_len}{2.5mm}

\begin{fmfgraph*}(80,60)
  \fmfkeep{qedcha-1a}
  \fmfleft{gamma}
  \fmfright{in,out}
  \fmf{plain}{out,va,v1,vb,vertex,vc,v2,vd,in}
  \fmf{photon}{vertex,gamma}
  \fmf{photon,tension=0}{v1,v2}
  \fmffreeze
  \fmf{dashes,right}{va,vb}
  \fmfdot{vertex,v1,v2,va,vb}
\end{fmfgraph*}
\end{fmffile}
\end{center}

Falls Feynman-Diagramme in Formel vorkommen sollen, empfiehlt es sich,
sie vorher in einer {\ttfamily sbox} zu definieren und dann sp�ter mit
{\ttfamily $\backslash$fmfreuse} aufzurufen:

% Use \sbox to prevent the diagrams from being displayed. We can now
% refer to them indidivually by their id (\fmfkeep)
\newsavebox{\charginodiagrams}
\sbox{\charginodiagrams}{
\begin{fmffile}{Feynman/Nichtplanar}
  \fmfset{thin}{.5pt}
  \fmfset{wiggly_len}{3mm}
  \fmfset{dash_len}{2.5mm}
  \fmfset{dot_size}{1thick}
  \fmfset{arrow_len}{2.5mm}

\begin{fmfgraph*}(80,60)
  \fmfkeep{qedcha-4a}
  \fmfleft{gamma}
  \fmfright{in,out}
  \fmf{plain}{out,va,v1,vertex,v2,vb,in}
  \fmf{photon}{vertex,gamma}
  \fmffreeze
  \fmf{dashes}{v1,vb}
  \fmf{photon}{v2,va}
  \fmfdot{vertex,v1,v2,va,vb}
\end{fmfgraph*}

\begin{fmfgraph*}(80,60)
  \fmfkeep{qedcha-4b}
  \fmfleft{gamma}
  \fmfright{in,out}
  \fmf{plain}{out,va,v1,vertex,v2,vb,in}
  \fmf{photon}{vertex,gamma}
  \fmffreeze
  \fmf{dashes}{v2,va}
  \fmf{photon}{v1,vb}
  \fmfdot{vertex,v1,v2,va,vb}
\end{fmfgraph*}
\end{fmffile}
}

\begin{equation}
  \fmfreuse{qedcha-4a} + \fmfreuse{qedcha-4b} = \text{endlich}
\end{equation}

Sch�ner sieht's vertikal zentriert aus.  Das geht mit {\ttfamily
  $\backslash$vcenter}, wenn man das Diagramm in eine {\ttfamily
  $\backslash$hbox} verpackt:

\newcommand{\fmfvcenter}[1]{\vcenter{\hbox{\fmfreuse{#1}}}}

\begin{equation}
  \fmfvcenter{qedcha-4a} + \fmfvcenter{qedcha-4b} = \text{endlich}
\end{equation}

Die modifizierte feynMF-Version erlaubt auch das Malen h�bscher
Countertermdiagramme mit dem neuen Befehl {\ttfamily $\backslash$fmfct}:

\newsavebox{\qedcounterterms}
\sbox{\qedcounterterms}{
\begin{fmffile}{Feynman/Counterterme}
  \fmfset{thin}{.5pt}
  \fmfset{wiggly_len}{3mm}
  \fmfset{dash_len}{2.5mm}
  \fmfset{dot_size}{1thick}
  \fmfset{arrow_len}{2.5mm}

\begin{fmfgraph*}(80,60)
  \fmfkeep{qedct-1a}
  \fmfleft{gamma}
  \fmfright{in,out}
  \fmf{plain}{out,v1,va,vertex,vb,v2,in}
  \fmf{photon}{vertex,gamma}
  \fmf{photon,tension=0}{v1,v2}
  \fmfdot{vertex,v2}
  \fmfct{v1}
\end{fmfgraph*}
\end{fmffile}
}

\begin{equation}
  \fmfvcenter{qedcha-1a} + \fmfvcenter{qedct-1a} = \text{endlich}
\end{equation}


\subsection{Erster Unterabschnitt}

Erster Unterabschnitt...

\subsection{Zweiter Unterabschnitt}

Zweiter Unterabschnitt...

\section{Etwas speziellere Grundlagen}

Etwas speziellere Grundlagen...

\section{Spezielle Grundlagen}

Spezielle Grundlagen...
